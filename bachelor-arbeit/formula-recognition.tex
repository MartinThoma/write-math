%!TEX root = thesis.tex
\chapter{Formula Recognition}\label{ch:FormulaRecognition}
\section{Nesting Structures}
One major issue of formula recognition are nesting structures:

\begin{itemize}
    \item \verb+\frac{[arbitrary math]}{[arbitrary math]}+
    \item \verb+\begin{pmatrix}[arbitrary math; structure with & ]\end{pmatrix}+
    \item \verb+\begin{align}[arbitrary math; structure with & ]\end{align}+
    \item \verb+\stackrel{[arbitrary math]}{[arbitrary math]}+
\end{itemize}

Another point that is special about handwritten math recognition compared to
natural language text are \textit{decorators}:

\begin{itemize}
    \item \verb+[symbol]_{[arbitrary math]}+
    \item \verb+[symbol]^{[arbitrary math]}+
\end{itemize}

\section{Segmentation}
I assume that writers finish one handwritten symbol before they start the next
symbol.