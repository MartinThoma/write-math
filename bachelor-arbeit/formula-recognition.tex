%!TEX root = thesis.tex
\chapter{Formula Recognition}\label{ch:FormulaRecognition}
\section{Nesting Structures}
One major issue of formula recognition are nesting structures:

\begin{itemize}
    \item \verb+\frac{[arbitrary math]}{[arbitrary math]}+
    \item \verb+\begin{pmatrix}[arbitrary math; structure with & ]\end{pmatrix}+
    \item \verb+\begin{align}[arbitrary math; structure with & ]\end{align}+
    \item \verb+\stackrel{[arbitrary math]}{[arbitrary math]}+
\end{itemize}

Another point that is special about handwritten math recognition compared to
natural language text are \textit{decorators}:

\begin{itemize}
    \item \verb+[symbol]_{[arbitrary math]}+
    \item \verb+[symbol]^{[arbitrary math]}+
\end{itemize}

A third problem that comes with formulas that consist of multiple symbols are
semantics: For single symbols, it is not possible to distinguish \verb+\Sigma+
from \verb+\sum+, so we don't expect an algorithm to be able to do so.
However, if we have a complete formula we might have enough context and thus
we want the recognition algorithm to be able to distinguish \verb+\Sigma+
from \verb+\sum+ and other pairs of similar symbols like \verb+\Pi+ and
\verb+\prod+ or \verb+\Omega+ and \verb+\Ohm+.

\section{Segmentation}
I assume that writers finish one handwritten symbol before they start the next
symbol.

Except for the following symbols, every symbol is composed of 4 strokes:

\begin{multicols}{2}
\begin{itemize}
    \item[12 lines] \verb+\Mundus+ - \Mundus
    \item[11 lines] \verb+\FAX+ - \FAX
    \item[9 lines] \verb+\sun+
    \item[8 lines] \verb+\idotsint+ - $\idotsint$
    \item[7 lines]
    \begin{itemize}
        \item \verb+\fax+ - \fax
        \item \verb+\Emailct+ - \Emailct
        \item \verb+\Letter+ - \Letter
        \item \verb+\EyesDollar+ - \EyesDollar
        \item \verb+\textreferencemark+ - \textreferencemark
        \item \verb+\neptune+ - %\neptune
        \item \verb+\ataribox+ - %\ataribox
    \end{itemize}
    \item[6 lines]
    \begin{itemize}
        \item \verb+\dots+ - \dots (wild points appear more often with single points)
        \item \verb+\vdots+ - \vdots
        \item \verb+\upuparrows+ - $\upuparrows$
        \item \verb+\Neptune+ - \Neptune
    \end{itemize}
    \item[5 lines]
    \begin{itemize}
        \item \verb+\Xi+ - $\Xi$
        \item \verb+\dotsint+ - $\dotsint$
        \item \verb+\idotsint+ - $\idotsint$
        \item \verb+\sqiint+ - $\sqiint$
        \item \verb+\nVDash+ - $\nVDash$
        \item \verb+\nLeftrightarrow+ - $\nLeftrightarrow$
        \item \verb+\boxtimes+ - $\boxtimes$
        \item \verb+\Smiley+, \verb+\smiley+, \verb+\Frowny+ - \Smiley, smiley, \Frowny
        \item \verb+\uranus+ - %\uranus
        \item \verb+\Uranus+ - \Uranus
        \item \verb+\neptune+ - %\neptune
        \item \verb+\textcurrency+ - \textcurrency
        \item \verb+\Lleftarrow+ - $\Lleftarrow$
        \item \verb+\Pi+ - $\Pi$
        \item \verb+\nexists+ - $\nexists$
        \item \verb+\updownarrow+ - $\updownarrow$
        \item \verb+\divideontimes+ $\divideontimes$
        \item \verb+\permil+ - (wasysym)
        \item \verb+\textpertenthousand+ - \textpertenthousand
        \item \verb+\textdiscount+ - \textdiscount
        \item \verb+\mathds{E}+ - $\mathds{E}$
        \item \verb+\mathds{F}+ - $\mathds{F}$
        \item \verb+\textsca+ - %\textsca
    \end{itemize}
\end{itemize}
\end{multicols}